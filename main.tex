\documentclass[conference]{IEEEtran}

\usepackage[utf8]{inputenc}

\title{Project Title}
\author{Name1 (\texttt{name1@cs.duke.edu}), Name2 (\texttt{name2@cs.duke.edu})}

\begin{document}

\maketitle

\begin{abstract}
    Abstract goes here
\end{abstract}
\section{Introduction}
The text is arbitrary; ignore the content.

ACM's consolidated article template, introduced in 2017, provides a
consistent \LaTeX\ style for use across ACM publications, and
incorporates accessibility and metadata-extraction functionality
necessary for future Digital Library endeavors. Numerous ACM and
SIG-specific \LaTeX\ templates have been examined, and their unique
features incorporated into this single new template\cite{Abril07}.

If you are new to publishing with ACM, this document is a valuable
guide to the process of preparing your work for publication. If you
have published with ACM before, this document provides insight and
instruction into more recent changes to the article template.

The ``\verb|acmart|'' document class can be used to prepare articles
for any ACM publication --- conference or journal, and for any stage
of publication, from review to final ``camera-ready'' copy, to the
author's own version, with {\itshape very} few changes to the source.

\section{Section 1}
Authors of any work published by ACM will need to complete a rights
form. Depending on the kind of work, and the rights management choice
made by the author, this may be copyright transfer, permission,
license, or an OA (open access) agreement.

Regardless of the rights management choice, the author will receive a
copy of the completed rights form once it has been submitted. This
form contains \LaTeX\ commands that must be copied into the source
document. When the document source is compiled, these commands and
their parameters add formatted text to several areas of the final
\section{Section 2}
\bibliographystyle{IEEEtran}
\bibliography{references}
\end{document}
